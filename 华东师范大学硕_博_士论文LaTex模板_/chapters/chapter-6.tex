\chapter{总结与展望}

\section{研究成果总结}

本文以 \textbf{MIL-STD-6016 / STANAG 5516} 为核心,围绕战术数据链信息标准的存储、查询与互操作展开了系统性的研究与实现。主要成果如下:

\begin{itemize}
  \item \textbf{数据库架构设计}:基于 MySQL~8.0 设计了 \texttt{spec}、\texttt{message}、\texttt{word}、\texttt{field}、\texttt{data\_item}、\texttt{concept} 等核心表结构,并通过 \texttt{concept\_binding} 机制实现了跨标准的概念映射,为后续的多版本对比与互操作提供支撑。
  \item \textbf{后端接口与绑定机制}:实现了 FastAPI 服务,包括 \texttt{/api/search}、\texttt{/api/word/search}、\texttt{/api/compare}、\texttt{/api/bind/field-to-di} 等核心接口,支持概念—字段—数据项的双向绑定和跨版本比较。
  \item \textbf{前端交互与可视化}:采用 React + shadcn/ui 开发了统一界面,实现搜索、比较与热门概念展示等功能,并提供了条件回显、弱错误提示与多浏览器兼容性优化。
  \item \textbf{测试与评估}:通过功能测试、压力测试与用户问卷,验证了系统在高并发与复杂检索条件下的稳定性与可用性。测试结果表明,系统能够在 1000 并发场景下保持较高吞吐与低延迟,具有良好的可扩展性。
\end{itemize}

综上,本文构建了一个贯通“标准化存储—智能搜索—跨标准比较—可视化展示”的完整实现链条,为后续的战术数据链互操作与仿真提供了技术参考。

\section{存在的不足}

尽管系统设计与实现达成了预期目标,但仍存在以下不足:

\begin{enumerate}
  \item \textbf{数据样本有限}:实验性导入的 MIL-STD-6016 字段数据规模约 50 万条,尚不足以覆盖全部 NATO 及扩展规范,缺乏跨多链(Link 11/22、TTNT)的大规模验证。
  \item \textbf{功能深度不足}:前端目前以表格展示为主,尚未集成复杂的图表分析与交互式态势展示;后端对冲突检测与一致性验证的逻辑仍偏基础。
  \item \textbf{安全与运维方面}:虽然实现了 RBAC 与 CORS 策略,但尚未覆盖全链路加密、KMI(密钥管理基础设施)等更高等级的安全要求。
\end{enumerate}

\section{未来研究方向}
后续的研究与应用可以从以下几个方面拓展:\cite{FFI_SDA_Link16_Space_2023,Redwire_SDA_Ship_Demo_2024,JAPCC_F35_TDLs_2024,JAPCC_Hosted_Payloads_2015}

\begin{itemize}
  \item \textbf{跨链互操作扩展}:引入 JREAP、SIMPLE、TTNT 等标准接口,进一步验证跨链数据转发与一致性处理机制。
  \item \textbf{智能化处理}:结合机器学习与大模型,对导入的 Link 16 数据链消息进行自动分类、冲突检测与语义推理,提升态势信息处理能力。
  \item \textbf{仿真与实验验证}:在实验平台中对多链融合能力进行实战化仿真,参考现有战术数据链仿真系统的设计思路,提升在复杂战场环境下的可验证性。
  \item \textbf{安全与加密机制}:探索基于区块链与分布式密钥管理的多链安全架构,增强跨域互操作环境下的可靠性与抗毁性。
\end{itemize}

通过以上拓展方向,未来系统有望在 \textbf{多链融合、智能分析、安全保障} 三个方面取得更高水平的突破,从而为新一代联合战术通信与仿真系统提供坚实的支撑。
