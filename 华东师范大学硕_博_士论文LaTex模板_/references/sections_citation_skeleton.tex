% ===== 可粘贴的 LaTeX 段落骨架(含 \cite{} 占位) =====

\chapter{第一章 绪论}
\section{1.1 研究背景与意义}
本节概述【写作提示】。\cite{NDIA_PMW101_2024,NDIA_PMW101_2022,DOTE_2022_MIDS_LVT,NAVAIR_MIDS_Overview,AviationWeek_SDA_LEO_2024,SDA_testing_OK_2023,MIL_STD_6016_Active_2024,L3Harris_MIDS_JTRS_2021}
\section{1.2 国内外研究现状}
本节概述【写作提示】。\cite{丁丁2019,马建强2020,程方昊2025,陈利玲2025,潘政昂2024,云超2025,wray_sheppard_1986_jtids_nav,ranger1996_jn,fried1978_taes,fried1984_navigation,doherty1988_jn,L3Harris_MIDS_LVT_2025}
\section{1.3 研究内容与目标}
本节概述【写作提示】。\cite{MITRE_Link16_Interoperability_2024,MIL_STD_6020_2017,MIL_STD_3011_2022,Collins_TTR_2021}
\section{1.4 研究方法与技术路线}
本节概述【写作提示】。\cite{SISO_STD_002_2006,CJCSI_6610_01F_2021,CJCSI_6610_01F_archive_2014,garciapena2021_navigation,musumeci_2016_nav_notch_wavelet,L3Harris_STT_KOR24A_2020}
\section{1.5 论文组织结构}
本节概述【写作提示】。\cite{Viasat_MOJO_Next_2025}

\chapter{第二章 相关工作}
\section{2.1 战术数据链(Tactical Data Link)概述}
本节概述【写作提示】。\cite{AFMAN_13_116_Vol1_2020,EverythingRF_Link16_Band,DLS_MIDS_JTRS_2021,BAE_Link16_Terminals_2025,Ultra_ADSI_2023}
\section{2.2 MIL-STD-6016 标准框架与特点}
本节概述【写作提示】。\cite{ASSIST_6016_2024,CJCSM_6235_01_2025,Ultra_MDLMS_2021}
\section{2.3 数据库设计与信息建模理论}
本节概述【写作提示】。\cite{Ultra_ADSI_2024_update,CurtissWright_LinkPRO_page,Kao_Robertson_MILCOM_2008,CurtissWright_TCG_BOSS_2025}
\section{2.4 前后端系统开发框架简述}
本节概述【写作提示】。\cite{SAIC_JRE_Overview_2021,DLS_TTR_2016,DLS_TTR_webpage,Collins_TTR_webpage,L3Harris_STT_KOR24A_Page_2025,CurtissWright_TCG_LinkPRO_2025}

\chapter{第三章 系统需求分析}
\section{3.1 功能需求}
本节概述【写作提示】。\cite{CurtissWright_TCG_HUNTR_2020}
\subsection{3.1.1 标准消息管理}
本小节内容【写作提示】。\cite{Chelton_Link16_Antennas_2022}
\subsection{3.1.2 字段与语义概念绑定}
本小节内容【写作提示】。\cite{AFCEA_Link16_Improvements_2022}
\subsection{3.1.3 多链路互操作支持}
本小节内容【写作提示】。\cite{Lekkakos_2008}
\subsection{3.1.4 数据库操作与维护}
本小节内容【写作提示】。\cite{Ho_2008}
\subsection{3.1.5 前端交互与可视化}
本小节内容【写作提示】。\cite{Kao_2008}
\subsection{3.1.6 仿真与验证接口}
本小节内容【写作提示】。\cite{Kagioglidis_2009}
\section{3.2 性能需求}
本节概述【写作提示】。\cite{Kee_2008}
\subsection{3.2.1 响应时间与并发能力}
本小节内容【写作提示】。\cite{baek2016_adhoc,baek2019_jsyst_timemirror,lee2018_jsyst,Spyridis_2010}
\subsection{3.2.2 数据吞吐量}
本小节内容【写作提示】。\cite{Kopp_Throughput_Enhanced_JTIDS_2006,Juarez_2025}
\subsection{3.2.3 可扩展性}
本小节内容【写作提示】。\cite{Viasat_STT_press_2020}
\subsection{3.2.4 可靠性与容错性}
本小节内容【写作提示】。\cite{Koromilas_2009,EverythingRF_STT}
\section{3.3 安全性需求}
本节概述【写作提示】。\cite{Collins_TTNT_immersion_2020}
\subsection{3.3.1 数据保密性与加密机制}
本小节内容【写作提示】。\cite{Euromids_2025_contract}
\subsection{3.3.2 访问控制与权限管理}
本小节内容【写作提示】。\cite{GovConWire_Euromids_2025}
\subsection{3.3.3 抗干扰与完整性保障}
本小节内容【写作提示】。\cite{musumeci_2014_ietrsn_pulseblank,borio_2013_ietspr_pulseblanking,houdzoumis2009_jn,wu_2016_taes_dme_wp,huo_2015_ieeecl_meb,huo_2015_comex_mixed_interference,mitch_2016_nav_chirp_geolocation,vandermerwe_2023_nav_mpanf,JCS_Directives_Library}
\subsection{3.3.4 容灾与备份}
本小节内容【写作提示】。\cite{CJCS_Manuals_Library}
\section{3.4 可扩展性需求}
本节概述【写作提示】。\cite{CJCS_Instructions_Library}
\subsection{3.4.1 标准演进适应性}
本小节内容【写作提示】。\cite{ASSIST_3011_2023}
\subsection{3.4.2 跨链路互操作扩展}
本小节内容【写作提示】。\cite{ASSIST_6020_2025}
\subsection{3.4.3 系统架构扩展性}
本小节内容【写作提示】。\cite{qin2013_gpssol}
\subsection{3.4.4 接口与开放性}
本小节内容【写作提示】。\cite{fried_loeliger1979_navigation}
\section{3.5 数据特征与处理需求}
本节概述【写作提示】。\cite{baruffa2013_jsps}
\subsection{3.5.1 数据来源与类型}
本小节内容【写作提示】。\cite{baek2016_jsac}
\subsection{3.5.2 数据结构特征}
本小节内容【写作提示】。\cite{baek2018_commag}
\subsection{3.5.3 数据处理需求}
本小节内容【写作提示】。\cite{jiang2019_sensors}
\subsection{3.5.4 数据存储与管理}
本小节内容【写作提示】。\cite{hegarty_1997_nav_interference}
\section{3.6 用户角色与交互需求}
本节概述【写作提示】。\cite{schnaufer_mcgraw_1997_waas}
\subsection{3.6.1 主要用户角色}
本小节内容【写作提示】。\cite{DefenseAdvancement_Ship_UAS_2025,anyaegbu_2008_nav_pulsed_mitigation}
\subsection{3.6.2 交互需求}
本小节内容【写作提示】。\cite{reid_2018_nav_leo}
\subsection{3.6.3 安全与权限控制}
本小节内容【写作提示】。\cite{huo_2015_electronics_letters_noise_est}

\chapter{第四章 数据库设计与实现}
\section{4.1 总体设计思路}
\subsection{4.1.1 设计目标}
本小节内容【写作提示】。
\subsection{4.1.2 设计原则}
本小节内容【写作提示】。
\subsection{4.1.3 总体架构}
本小节内容【写作提示】。
\section{4.2 数据表结构与 ER 模型}
\subsection{4.2.1 建模原则}
本小节内容【写作提示】。
\subsection{4.2.2 核心实体与关系}
本小节内容【写作提示】。
\subsection{4.2.3 主外键与约束设计}
本小节内容【写作提示】。
\subsection{4.2.4 版本管理与审计扩展}
本小节内容【写作提示】。
\subsection{4.2.5 典型 DDL 片段(示例)}
本小节内容【写作提示】。
\subsection{4.2.6 索引策略与典型查询}
本小节内容【写作提示】。
\subsection{4.2.7 与 ER 图的对应关系}
本小节内容【写作提示】。
\section{4.3 数据导入与清洗(CSV/Excel)}
\subsection{4.3.1 数据导入流程}
本小节内容【写作提示】。
\subsection{4.3.2 数据清洗策略}
本小节内容【写作提示】。
\subsection{4.3.3 批量处理与效率优化}
本小节内容【写作提示】。
\section{4.4 索引与查询优化}
\subsection{4.4.1 索引设计原则}
本小节内容【写作提示】。
\subsection{4.4.2 典型查询优化}
本小节内容【写作提示】。
\subsection{4.4.3 查询优化策略}
本小节内容【写作提示】。
\subsection{4.4.4 性能评估}
本小节内容【写作提示】。
\section{4.5 数据一致性与完整性保障}
\subsection{4.5.1 事务一致性与并发控制}
本小节内容【写作提示】。
\subsection{4.5.2 完整性约束与规则定义}
本小节内容【写作提示】。
\subsection{4.5.3 跨链互操作一致性}
本小节内容【写作提示】。
\subsection{4.5.4 实时监控与异常处理}
本小节内容【写作提示】。
\subsection{4.5.5 版本管理与可追溯性}
本小节内容【写作提示】。

\chapter{第五章 系统架构与接口设计}
\section{5.1 系统总体架构}
\subsection{5.1.1 分层设计}
本小节内容【写作提示】。
\subsection{5.1.2 数据流转}
本小节内容【写作提示】。
\section{5.2 后端 FastAPI 服务设计}
\section{5.3 前端 React + shadcn/ui 界面设计}
\subsection{5.3.1 设计目标与原则}
本小节内容【写作提示】。
\subsection{5.3.2 信息架构与交互流程}
本小节内容【写作提示】。
\subsection{5.3.3 主要组件与样式}
本小节内容【写作提示】。
\subsection{5.3.4 状态管理与数据流}
本小节内容【写作提示】。
\subsection{5.3.5 接口映射与契约}
本小节内容【写作提示】。
\subsection{5.3.6 用户体验优化}
本小节内容【写作提示】。
\subsection{5.3.7 演进与扩展}
本小节内容【写作提示】。
\section{5.4 API 接口与调用流程}
\subsection{5.4.1 参数约束与契约管理}
本小节内容【写作提示】。
\subsection{5.4.2 调用性能与优化}
本小节内容【写作提示】。
\section{5.5 安全与权限控制}
\subsection{5.5.1 跨域与 API 安全}
本小节内容【写作提示】。
\subsection{5.5.2 数据库安全}
本小节内容【写作提示】。
\subsection{5.5.3 审计与追踪}
本小节内容【写作提示】。
\subsection{5.5.4 部署与运行时安全}
本小节内容【写作提示】。
\subsection{5.5.5 未来扩展}
本小节内容【写作提示】。

\chapter{第六章 系统实现与功能展示}
\section{6.1 数据库初始化与数据导入}
\subsection{6.1.1 数据库初始化}
本小节内容【写作提示】。
\subsection{6.1.2 CSV/Excel 批量导入}
本小节内容【写作提示】。
\subsection{6.1.3 数据清洗与一致性保障}
本小节内容【写作提示】。
\subsection{6.1.4 导入流程图}
本小节内容【写作提示】。
\section{6.2 核心功能模块实现}
\section{6.3 搜索与比较功能演示}
\subsection{6.3.1 搜索演示}
本小节内容【写作提示】。
\subsection{6.3.2 比较演示}
本小节内容【写作提示】。
\section{6.4 前端交互与可视化效果}
\section{6.5 系统运行环境与部署}
\subsection{6.5.1 运行环境}
本小节内容【写作提示】。
\subsection{6.5.2 部署方式}
本小节内容【写作提示】。
\subsection{6.5.3 运行效果}
本小节内容【写作提示】。

\chapter{第七章 系统测试与性能分析}
\section{7.1 测试方案与方法}
\section{7.2 功能测试结果}
\section{7.3 性能与压力测试}
本节概述【写作提示】。\cite{schneckenburger_2018_nav_multipath}
\section{7.4 用户体验评估}

\chapter{第八章 总结与展望}
\section{8.1 研究成果总结}
\section{8.2 存在的不足}
\section{8.3 未来研究方向}
本节概述【写作提示】。\cite{FFI_SDA_Link16_Space_2023,Redwire_SDA_Ship_Demo_2024,JAPCC_F35_TDLs_2024,JAPCC_Hosted_Payloads_2015}
