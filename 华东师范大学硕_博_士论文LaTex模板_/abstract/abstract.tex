\chapter*{\xiaosan\heiti{摘\quad 要}}
\addcontentsline{toc}{chapter}{摘要}

随着信息化作战和多域作战的发展,战术信息链成为联合作战中信息共享、指挥控制的技术手段,以Link16为代表的信息链技术,基于MIL-STD-6016标准,以J系列报文为核心进行信息跨平台、跨军种的互联共享。然而在现有系统中仍存在标准化描述、语义互操作、系统扩展性等方面的短板,难以满足多域协同作战和智能化作战的需求。

针对上述问题,本研究在战术数据链信息标准数据库系统设计与实现方面取得了四个主要的成果:

首先,引入 Common Data Model(CDM)四层语义映射法,构建了概念层、协议层、消息层和字段层组成的完整互操作体系。通过语义绑定与映射规则统一建模,实现了 MIL-STD-6016、MAVLink 与 MQTT 等异构协议间的字段级语义对齐,映射正确率也大幅度提升。

其次,构建了基于多模态深度学习的标准文档智能解析框架。系统融合 PyMuPDF、pdfplumber、Camelot 与 Tesseract OCR 等多引擎,通过层次化表格识别实现标准文档的自动化结构化抽取,解析精度达 99.8\%,位长计算误差控制在 0.02\% 以内。

第三,设计了分布式微服务化数据管理体系,采用 MySQL 主从复制与 Redis 缓存协同机制,引入 Elasticsearch 实现全文索引。系统在 1000 并发访问下平均响应延迟降至 280 ms,缓存命中率达到 91.2\%。

最后,提出了可插拔的协议适配器架构,支持不同标准模块的动态加载与语义继承。系统在 1500 条/秒处理速率下保持 98.9\% 的一致性验证通过率,显著优于手工适配方案。

实验验证表明,系统在百万级数据规模下实现毫秒级查询响应,搜索准确率维持在 95\% 以上,用户满意度达 4.1/5.0。本研究为战术数据链信息标准化与语义融合提供了创新的技术路径,所构建的架构在多链融合、战场态势共享以及装备互操作性测试等应用场景中展现出重要的工程实践价值。
\vspace{0.5cm}
% \hspace{-1cm}

\sihao{\heiti{关键词:}}\xiaosi{战术数据链;MIL-STD-6016;微服务架构;语义互操作;文档自动化处理;数据库建模}
