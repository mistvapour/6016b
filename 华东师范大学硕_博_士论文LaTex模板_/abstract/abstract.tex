\chapter*{\xiaosan\heiti{摘\quad 要}}
\addcontentsline{toc}{chapter}{摘要}

随着信息化战争与多域作战的不断深化,战术数据链(Tactical Data Link, TDL)已成为实现联合作战中信息共享与指挥控制的关键支撑技术。其中,{Link16} 作为典型代表,基于 MIL-STD-6016 标准,以 J 系列报文为核心,实现了跨平台、跨军种的信息互联。然而,现有系统在标准化建模、语义互操作及系统可扩展性等方面仍存在不足,难以满足多域协同和智能化作战的需求。

首先,本文在深入分析 MIL-STD-6016 标准及其 J 系列消息体系的基础上,提出了一种面向战术数据链的信息标准数据库建模方法。通过对消息体系进行层次化解析,实现了“消息—字—字段—数据项”的结构化建模与统一管理,并引入语义概念绑定机制,建立跨标准的语义映射模型,为多链互操作提供了语义层支撑。

其次,本文设计并实现了一种基于云原生与微服务架构的系统方案。系统将单体后端拆分为 PDF 处理、语义互操作、CDM 建模、统一导入等独立服务模块,并通过 API 网关实现统一接入、认证授权与流量控制。底层集成 MySQL、Redis、RabbitMQ 与 MinIO 等组件,支持高并发访问与最终一致性保障。同时,系统采用 FastAPI 与 React 技术栈开发,通过 Docker 与 Kubernetes 实现容器化部署与自动化运维(CI/CD)。

最后,本文对系统进行了功能与性能验证。实验结果表明,系统在百万级数据规模下可实现毫秒级响应,平均延迟小于 200 ms,缓存命中率超过 85\%,并在高并发场景下保持 99\% 的稳定性。结果验证了该系统在标准化建模、语义互操作与架构扩展性方面的有效性。

本文的研究成果为战术数据链的信息标准化与语义融合提供了新的技术思路,所提出的数据库与微服务架构在多链融合、态势共享与装备互操作测试中具有重要的工程应用价值。

\vspace{0.5cm}
% \hspace{-1cm}
\sihao{\heiti{关键词:}}\xiaosi{战术数据链;MIL-STD-6016;微服务架构;语义互操作;自动化部署;数据库建模}
